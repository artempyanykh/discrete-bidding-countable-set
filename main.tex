\documentclass[12pt, draft]{extarticle}

\usepackage[a4paper]{geometry}%
\usepackage[utf8]{inputenc}%
\usepackage[english,russian]{babel}%
\usepackage{amsmath}%
\usepackage{amssymb}%

\begin{document}

\title{Многошаговая модель биржевых торгов с элементами переговоров: расширение
  на случай счетного множества состояний%
}%
\author{%
  Артем Пьяных\\
  Московский университет им.М.В.Ломоносова\\
  Факультет вычислительной математики и кибернетики\\
  \texttt{artem.pyanykh@gmail.com}%
}%
\maketitle

\begin{abstract}
  Рассматривается упрощенная модель финансового рынка, в которой два игрока
  ведут торги за однотипные акции в течение $n$ последовательных шагов. Игрок 1
  (инсайдер) информирован о настоящей ликвидной цене акции, которая может
  принимать любое значение из $\mathbb{Z}_+$. В то же время Игрок 2 знает только
  вероятностное распределение $p$ цены акции. На каждом шаге торгов игроки
  делают целочисленные ставки. Игрок, предложивший б\'{о}льшую ставку покупает у
  другого акцию по цене, равной выпуклой комбинации предложенных ставок.
  Получено решение игры неограниченной продолжительности для распределений $p$ с
  конечным вторым моментом.

  \textit{Ключевые слова}: многошаговые игры, асимметричная информация,
  инсайдерская торговля
\end{abstract}

Основной текст статьи начинается здесь. Вот для интереса формула:
\begin{equation}
  \lim_{n\rightarrow\infty} \left( 1 + \frac{1}{n} \right)^n = e.
\end{equation}

\end{document}
