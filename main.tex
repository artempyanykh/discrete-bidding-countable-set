\documentclass[12pt, draft]{extarticle}

\usepackage[a4paper]{geometry}%

\usepackage{pscyr}

\usepackage[utf8]{inputenc}%
\usepackage[T2A]{fontenc}
\usepackage[english,russian]{babel}%

\usepackage{amsmath}%
\usepackage{amssymb}%
\usepackage{amsthm}%
\usepackage{textcomp}%

\newtheorem{definition}{Определение}

% Visuals
\usepackage{indentfirst}
\frenchspacing

\usepackage{fancyhdr}
\lhead{}
\chead{Модель биржевых торгов со счетным множеством состояний}
\rhead{}
\lfoot{}
\cfoot{\thepage}
\rfoot{}
\renewcommand{\headrulewidth}{0.8pt}
\renewcommand{\footrulewidth}{0.8pt}
\pagestyle{fancy}

% \renewenvironment{abstract}{%
%   \hfill\begin{minipage}{0.95\textwidth}
%     \rule{\textwidth}{1pt}\small}
%   {\normalsize\par\noindent\rule{\textwidth}{1pt}\end{minipage}}
% 
% \makeatletter
% \renewcommand\@maketitle{%
%   \newpage
%   \null
%   \vskip 2em%
%   \begin{center}%
%     \let \footnote \thanks
%     {\Large \@title \par}%
%     \vskip 1.5em%
%     {\normalsize
%       \lineskip .5em%
%       \begin{tabular}[t]{c}%
%         \@author
%       \end{tabular}\par}%
%     \vskip 1em%
%     {\normalsize \@date}%
%   \end{center}%
%   \par
%   \vskip 1.5em}
% \makeatother

\begin{document}

\title{Многошаговая модель биржевых торгов с элементами переговоров: расширение
  на случай счетного множества состояний%
}%
\author{%
  Артем Пьяных%
  \thanks{%
    Исследование выполнено при финансовой поддержке РФФИ в рамках научного
    проект \textnumero 16-01-00353a.%
  }\\
  Московский университет им. М.В. Ломоносова\\
  Факультет вычислительной математики и кибернетики\\
  \texttt{artem.pyanykh@gmail.com}%
}%
\maketitle

\begin{abstract}
  Рассматривается упрощенная модель финансового рынка, на котором два игрока
  ведут торги за однотипные акции в течение $n$ шагов. Игрок 1 (инсайдер)
  информирован о настоящей ликвидной цене акции, которая может принимать любое
  значение из $\mathbb{Z}_+$. В то же время Игрок 2 знает только вероятностное
  распределение $p$ цены акции. На каждом шаге торгов игроки делают
  целочисленные ставки. Игрок, предложивший б\'{о}льшую ставку покупает у
  другого акцию по цене, равной выпуклой комбинации предложенных ставок.
  Получено решение игры неограниченной продолжительности для распределений $p$ с
  конечной дисперсией.

  \textbf{Ключевые слова}: многошаговые игры, асимметричная информация,
  инсайдерская торговля.
\end{abstract}

\section{Введение}
\label{sec:intro}

В данной работе рассматривается упрощенная модель финансового рынка, на котором
два игрока ведут торги за однотипные акции на протяжении $n \leqslant \infty$
шагов. Перед началом торгов случайный ход определяет цену акции $s \in S$ на
весь период торгов в соответствии с вероятностным распределением $p = (p_x, \; x
\in S)$. Выбранная цена сообщается Игроку 1 (инсайдеру). Игрок 2 при этом знает
только вероятностное распределение $p$ и не осведомлен о настоящем значении
цены. На каждом шаге торгов игроки одновременно и независимо назначают некоторую
цену за акцию. Игрок, сделавший большую ставку, покупает акцию у другого;
если ставки равны, сделки не происходит. Задачей игроков является максимизация
стоимости итогового портфеля, состоящего из некоторого числа акций и суммы
денег. Данное описание считается известным обоим игрокам.

Модель, в которой цена акции может принимать только значения $0$ и $m$, была
рассмотрена в \cite{bib:domansky07}. Задача сводится к анализу антагонистической
повторяющейся игры с неполной информацией, как описано в \cite{bib:aumann}. В
рамках данной модели неосведомленный Игрок 2 использует историю ставок Игрока 1
для пересчета апостериорных вероятностей значения цены акции. Остюда, задачей
Игрока 1 является поиск стратегии, которая позволит ему контролировать
последовательность апостериорных вероятностей таким образом, чтобы Игрок 2 как
можно дольше не мог догадаться о настоящем значении цены. В
\cite{bib:domansky07} показано, что последовательность верхних значений
$n$-шаговых игр ограничена, что позволило определить игру с бесконечным
количеством шагов, для которой были найдены оптимальные стратегии игроков и
значение.
%
Для игр с конечным количеством шагов аналитические решения получены только в
ограниченном количестве случаев: в \cite{bib:sandomirskaya12} получено решение
одношаговой игры при произвольном натуральном значения $m$; в \cite{bib:kreps09}
получено решение $n$-шаговых игр при $m \leqslant 3$. Аналитическое решение игр
с конечным количеством шагов в общем случае остается открытой проблемой.
%
В работе \cite{bib:domansky11} рассмотрено обобщение модели на случай, когда
цена акции может принимать любое значение $s \in S = \mathbb{Z}_+$. Показано,
что если $\mathbb{D} p < \infty$, то последовательность верхних значений игры
ограничена, что снова позволяет определить игру с бесконечным количеством шагов,
для которой авторами найдено решение.

В работах \cite{bib:domansky07, bib:domansky11} сделка осуществляется по цене,
равной наибольшей предложенной ставке. Можно, однако, рассмотреть другой
механизм, предложенный в \cite{bib:chatterjee83}, и положить цену сделки равной
выпуклой комбинации предложенных ставок с коэффициентом $\beta \in [0,1]$, т.е.
если игроками были сделаны ставки $p_1 \neq p_2$, то акция будет продана по цене
$\beta \max(p_1, p_2) + (1-\beta) \min(p_1, p_2)$. Фактически в
\cite{bib:domansky07, bib:domansky11} коэффициент $\beta$ равен $1$. Обобщение
модели с двумя возможными значениями цены на случай произвольного $\beta$ было
проведено в \cite{bib:pyanykh16}. В данной работе обобщение на случай
произвольного $\beta$ проведено для модели со счетным множеством возможных
значений цены.

\section{Постановка задачи}
\label{sec:problem-statement}

Пусть множество состояний рынка $S = \mathbb{Z}_+$. Перед началом игры случай
выбирает $s \in S$ в соответствии с вероятностным распределением $p = (p_x, \; x
\in S)$. На каждом шаге игры $t = \overline{1,n}, \; n \leqslant \infty$ игроки
делают ставки $i_t \in I, \, j_t \in J$, где $I = J = \mathbb{Z}_+$. Выплата
Игроку 1 в состоянии $s$ равна
\begin{equation*}
  % \label{eq:ps:stage-payoff}
  a^s(i_t, j_t) =
  \begin{cases}
    (1-\beta) i_t + \beta j_t - s, &\; i_t < j_t, \\
    0, &\; i_t = j_t, \\
    s - \beta i_t - (1-\beta)j_t, &\; i_t > j_t.
  \end{cases}
\end{equation*}

Обозначим через $\Delta(X)$ множество вероятностных распределений над множеством
$X$.
\begin{definition}
  Стратегией Игрока 1 является последовательность ходов $\sigma = (\sigma_1,
  \ldots, \sigma_n)$, где $\sigma_t: S \times I^{t-1} \rightarrow \Delta(I)$.
\end{definition}

\begin{definition}
  Стратегией Игрока 2 является последовательность ходов $\tau = (\tau_1, \ldots,
  \tau_n)$, где $\tau_t: I^{t-1} \rightarrow \Delta(J)$.
\end{definition}

То есть, Игрок 1 на каждом шаге игры рандомизирует свои действия в зависимости
от состояния рынка $s$ и истории ставок. Игрок 2 в свою очередь, не имея
информации о состоянии рынка $s$, опирается только на историю ставок инсайдера.

\begin{thebibliography}{99}
% \bibitem{bib:demeyer02}%
%   De Meyer B., Saley H. \emph{On the strategic origin of Brownian motion in
%     finance} // International Journal of Game Theory. 2002. V. 31. P. 285--319.

\bibitem{bib:domansky07}%
  Domansky V. \emph{Repeated games with asymmetric information and random price
    fluctuations at finance markets} // International Journal of Game Theory.
  2007. V. 36(2). P. 241--257.

\bibitem{bib:aumann}%
  Aumann R.J., Maschler M.B. \emph{Repeated Games with Incomplete Information}.
  The MIT Press, Cambridge, London.

\bibitem{bib:sandomirskaya12}%
  Сандомирская М.С., Доманский В.К. \emph{Решение одношаговой игры биржевых
    торгов с неполной информацией} // Математическая теория игр и ее приложения.
  2012. 4. №1. С. 32-54.

\bibitem{bib:kreps09}%
  Крепс В.Л. \emph{Повторяющиеся игры, моделирующие биржевые торги, и возвратные
    последовательности} // Известия РАН. Теория и системы управления. 2009. № 4.
  С. 109--120.

\bibitem{bib:domansky11}%
  Доманский В.К., Крепс В.Л. \emph{Теоретико-игровая модель биржевых торгов:
    стратегические аспекты формирования цен на фондовых рынках} // Журнал Новой
  экономической ассоциации. 2011. Вып. 11. C. 39–-62.

\bibitem{bib:chatterjee83}%
  Chatterjee K., Samuelson W. \emph{Bargaining under Incomplete Information} //
  Operations Research. 1983. V. 31. N. 5. P. 835--851.

% \bibitem{bib:pyanykh14}%
%   Пьяных А.И. \emph{Об одной модификации модели биржевых торгов с инсайдером} //
%   Математическая теория игр и её приложения. 2014. 6. № 4. C. 68--84.

\bibitem{bib:pyanykh16}%
  Пьяных А.И. \textit{Многошаговая модель биржевых торгов с асимметричной
    информацией и элементами переговоров} // Вестн. Моск. ун-та. Сер.15. Вычисл.
  матем. и киберн. 2016. №1. С. 34—40.
\end{thebibliography}

\end{document}
