\documentclass[12pt, draft]{extarticle}

\usepackage[a4paper]{geometry}%

\usepackage{pscyr}

\usepackage[utf8]{inputenc}%
\usepackage[T2A]{fontenc}
\usepackage[english,russian]{babel}%

\usepackage{amsmath}%
\usepackage{amssymb}%
\usepackage{amsthm}%
\usepackage{amsopn}%
\usepackage{amsfonts}
\usepackage{textcomp}%

\newtheorem{definition}{Определение}
\newtheorem{lemma}{Лемма}
\newtheorem{theorem}{Теорема}

% Visuals
\usepackage{indentfirst}
\frenchspacing

\usepackage{fancyhdr}
\lhead{}
\chead{Модель биржевых торгов со счетным множеством состояний}
\rhead{}
\lfoot{}
\cfoot{\thepage}
\rfoot{}
\renewcommand{\headrulewidth}{0.8pt}
\renewcommand{\footrulewidth}{0.8pt}
\pagestyle{fancy}

% \renewenvironment{abstract}{%
%   \hfill\begin{minipage}{0.95\textwidth}
%     \rule{\textwidth}{1pt}\small}
%   {\normalsize\par\noindent\rule{\textwidth}{1pt}\end{minipage}}
% 
% \makeatletter
% \renewcommand\@maketitle{%
%   \newpage
%   \null
%   \vskip 2em%
%   \begin{center}%
%     \let \footnote \thanks
%     {\Large \@title \par}%
%     \vskip 1.5em%
%     {\normalsize
%       \lineskip .5em%
%       \begin{tabular}[t]{c}%
%         \@author
%       \end{tabular}\par}%
%     \vskip 1em%
%     {\normalsize \@date}%
%   \end{center}%
%   \par
%   \vskip 1.5em}
% \makeatother
\newcommand{\overbar}[1]%
{\mkern 1.5mu\overline{\mkern-1.3mu#1\mkern-1.3mu}\mkern 1.3mu}

\newcommand{\s}{\ensuremath{s}}
\newcommand{\p}{\ensuremath{\overbar{p}}}
\newcommand{\q}{\ensuremath{\overbar{q}}}
\DeclareMathOperator{\E}{\mathbb{E}}
\DeclareMathOperator{\D}{\mathbb{D}}
\DeclareMathOperator{\Z}{\mathbb{Z}}
\newcommand{\G}[1][n]{\ensuremath{G_{#1}}}
\newcommand{\K}[1][n]{\ensuremath{K_{#1}}}
\DeclareMathOperator{\FPStrategies}{\Sigma}
\DeclareMathOperator{\SPStrategies}{\mathrm{T}}
\newcommand{\V}[1][n]{\ensuremath{V_{#1}}}
\newcommand{\High}[1][\ensuremath{\infty}]{\ensuremath{H_{#1}}}
\newcommand{\sigmav}{\ensuremath{\overbar{\sigma}}}
\newcommand{\tauv}{\ensuremath{\overbar{\tau}}}

\begin{document}

\title{Многошаговая модель биржевых торгов с элементами переговоров: расширение
  на случай счетного множества состояний%
}%
\author{%
  Артем Пьяных%
  \thanks{%
    Исследование выполнено при финансовой поддержке РФФИ в рамках научного
    проект \textnumero 16-01-00353a.%
  }\\
  Московский университет им. М.В. Ломоносова\\
  Факультет вычислительной математики и кибернетики\\
  \texttt{artem.pyanykh@gmail.com}%
}%
\maketitle

\begin{abstract}
  Рассматривается упрощенная модель финансового рынка, на котором два игрока
  ведут торги за однотипные акции в течение $n$ шагов. Игрок 1 (инсайдер)
  информирован о настоящей ликвидной цене акции, которая может принимать любое
  значение из $\Z_+$. В то же время игрок 2 знает только вероятностное
  распределение $\p$ цены акции. На каждом шаге торгов игроки делают
  целочисленные ставки. Игрок, предложивший б\'{о}льшую ставку покупает у
  другого акцию по цене, равной выпуклой комбинации предложенных ставок.
  Получено решение игры неограниченной продолжительности для распределений $\p$
  с конечной дисперсией.

  \textbf{Ключевые слова}: многошаговые игры, асимметричная информация,
  инсайдерская торговля.
\end{abstract}

\section{Введение}
\label{sec:intro}

В данной работе рассматривается упрощенная модель финансового рынка, на котором
два игрока ведут торги за однотипные акции на протяжении $n \leqslant \infty$
шагов. Перед началом торгов случайный ход определяет цену акции $\s \in S$ на
весь период торгов в соответствии с вероятностным распределением $\p = (p_s, \;
s \in S)$. Выбранная цена сообщается игроку 1 (инсайдеру). Игрок 2 при этом
знает только вероятностное распределение $\p$ и не осведомлен о настоящем
значении цены. На каждом шаге торгов игроки одновременно и независимо назначают
некоторую цену за акцию. Игрок, сделавший большую ставку, покупает акцию у
другого; если ставки равны, сделки не происходит. Задачей игроков является
максимизация стоимости итогового портфеля, состоящего из некоторого числа акций
и суммы денег. Данное описание считается известным обоим игрокам.

Модель, в которой цена акции может принимать только значения $0$ и $m$, была
рассмотрена в \cite{bib:domansky07}. Задача сводится к анализу антагонистической
повторяющейся игры с неполной информацией, как описано в \cite{bib:aumann}. В
рамках данной модели неосведомленный игрок 2 использует историю ставок игрока 1
для пересчета апостериорных вероятностей значения цены акции. Остюда, задачей
игрока 1 является поиск стратегии, которая позволит ему контролировать
последовательность апостериорных вероятностей таким образом, чтобы игрок 2 как
можно дольше не мог догадаться о настоящем значении цены. В
\cite{bib:domansky07} показано, что последовательность верхних значений
$n$-шаговых игр ограничена, что позволило определить игру с бесконечным
количеством шагов, для которой были найдены оптимальные стратегии игроков и
значение.
%
Для игр с конечным количеством шагов аналитические решения получены только в
ограниченном количестве случаев: в \cite{bib:sandomirskaya12} получено решение
одношаговой игры при произвольном натуральном значения $m$; в \cite{bib:kreps09}
получено решение $n$-шаговых игр при $m \leqslant 3$. Аналитическое решение игр
с конечным количеством шагов в общем случае остается открытой проблемой.
%
В работе \cite{bib:domansky11} рассмотрено обобщение модели на случай, когда
цена акции может принимать любое значение $\s \in S = \mathbb{Z}_+$. Показано,
что если $\mathbb{D} \p < \infty$, то последовательность верхних значений игры
ограничена, что снова позволяет определить игру с бесконечным количеством шагов,
для которой авторами найдено решение.

В работах \cite{bib:domansky07, bib:domansky11} сделка осуществляется по цене,
равной наибольшей предложенной ставке. Можно, однако, рассмотреть другой
механизм, предложенный в \cite{bib:chatterjee83}, и положить цену сделки равной
выпуклой комбинации предложенных ставок с коэффициентом $\beta \in [0,1]$, т.е.
если игроками были сделаны ставки $p_1 \neq p_2$, то акция будет продана по цене
$\beta \max(p_1, p_2) + (1-\beta) \min(p_1, p_2)$. Фактически в
\cite{bib:domansky07, bib:domansky11} коэффициент $\beta$ равен $1$. Обобщение
модели с двумя возможными значениями цены на случай произвольного $\beta$ было
проведено в \cite{bib:pyanykh16}. В данной работе обобщение на случай
произвольного $\beta$ проведено для модели со счетным множеством возможных
значений цены.

\section{Постановка задачи}
\label{sec:problem-statement}

Пусть множество состояний рынка $S = \Z_+$. Перед началом игры случай выбирает
состояние рынка $\s \in S$ в соответствии с вероятностным распределением $\p =
(p_s, \; s \in S)$. На каждом шаге игры $t = \overline{1,n}, \; n \leqslant
\infty$ игроки делают ставки $i_t \in I, \, j_t \in J$, где $I = J = \Z_+$.
Выплата игроку 1 в состоянии $s$ равна
\begin{equation*}
  % \label{eq:ps:stage-payoff}
  a^s(i_t, j_t) =
  \begin{cases}
    (1-\beta) i_t + \beta j_t - s, &\; i_t < j_t, \\
    0, &\; i_t = j_t, \\
    s - \beta i_t - (1-\beta)j_t, &\; i_t > j_t.
  \end{cases}
\end{equation*}

Обозначим через $\Delta(X)$ множество вероятностных распределений над $X$.
\begin{definition}
  Стратегией игрока 1 является последовательность ходов $\sigmav = (\sigma_1,
  \ldots, \sigma_n)$, где $\sigma_t: S \times I^{t-1} \rightarrow \Delta(I)$.
  Множество стратегий игрока 1 обозначим $\FPStrategies$.
\end{definition}

\begin{definition}
  Стратегией игрока 2 является последовательность ходов $\tauv = (\tau_1,
  \ldots, \tau_n)$, где $\tau_t: I^{t-1} \rightarrow \Delta(J)$. Множество
  стратегий игрока 2 обозначим $\SPStrategies$.
\end{definition}

То есть, игрок 1 на каждом шаге игры рандомизирует свои действия в зависимости
от состояния рынка $s$ и истории ставок. Игрок 2 в свою очередь, не имея
информации о состоянии рынка $s$, опирается только на историю ставок инсайдера.

Будем считать, что игроки обладают неограниченными запасами рисковых и
безрисковых активов, т.е. торги не могут закончиться по причине того, что у
одного из игроков закончаться деньги или акции. Кроме того, будем считать, что в
начальный момент времени оба игрока имеют нулевые портфели.

\begin{definition}
  При использовании игроком 1 стратегии $\sigma$, игроком 2 -- стратегии $\tau$,
  ожидаемый выигрыш игрока 1 равен
  \begin{equation*}
    \K(\p, \sigmav, \tauv) =
    \E_{(\p, \sigmav, \tauv)} \sum_{t=1}^n a^s(i_t, j_t),
  \end{equation*}
  где математическое ожидание берется по мере, индуцированной $\p$, $\sigmav$ и
  $\tauv$.
\end{definition}
Заданную таким образом игру обозначим $\G(\p)$.

\begin{definition}
  Если для некоторых $\sigmav^* \in \FPStrategies,$ $\tauv^* \in \SPStrategies$
  выполняется
  \begin{equation*}
    \sup_{\sigmav \in \FPStrategies} \inf_{\tauv \in \SPStrategies} \K(\p, \sigmav, \tauv) =
    \K(\p, \sigmav^*, \tauv^*) =
    \inf_{\tauv \in \SPStrategies} \sup_{\sigmav \in \FPStrategies} \K(\p, \sigmav, \tauv) = 
    \V(\p),
  \end{equation*}
  то игра $\G(\p)$ имеет значение $\V(\p)$, а стратегии $\sigmav^*$ и $\tauv^*$
  являются оптимальными.
\end{definition}

Следуя \cite{bib:domansky11}, опишем рекурсивную структуру игры $\G(\p)$.
Представим стратегию игрока 1 в виде $\sigmav = (\sigma, \sigmav^i, i \in I)$,
где $\sigma$ --- ход игрока на первом шаге, а $\sigmav^i$ --- стратегия в игре
продолжительности $n-1$ в зависимости от ставки $i$ на первом шаге. Аналогично,
стратегию игрока 2 представим в виде $\tauv = (\tau, \tauv^i, \; i \in I)$.
%
Далее, обозначим $q_i$ полную вероятность, с которой игрок 1 делает ставку $i
\in I$, и $\q = (q_i, \; i \in I)$ --- соответствующее распределение. Также
обозначим $p_s^i$ апостериорную вероятность состояния $s$ в зависимости от
ставки $i$ игрока 1 и $\p^i = (p_s^i, s \in S)$ --- соответствующее
апостериорное распределение. Тогда для значения выигрыша будет справедлива
формула
\begin{equation*}
  \K[n](\p, \sigmav, \tauv) =
  \K[1](\p, \sigma, \tau) +
  \sum_{i \in I} q_i \K[n-1](\p^i, \sigmav^i, \tauv^i).
\end{equation*}
Таким образом, определить стратегию в игре произвольной продолжительности можно,
задав ход игрока для любого значения $\p$.

\section{Оценки на выигрыш в игре $\mathbf{\G[\infty](\p)}$}
\label{sec:payoff-bounds}

Следуя \cite{bib:domansky11} рассмотрим чистую стратегию $\tauv^k$ игрока 2,
определенную следующим образом:
\begin{equation*}
  \tau^k_1 = k, \quad
  t^k_t = \begin{cases}
    j_{t-1}, &\; i_{t-1} < j_{t-1},\\
    j_t, &\; i_{t-1} = j_{t-1},\\
    j_{t+1}, &\; i_{t-1} > j_{t-1}.
  \end{cases}
\end{equation*}
При использовании этой стратегии игрок 2 делает ставку равную $k$ на первом
шаге, а далее подражает инсайдеру. Введем обозначение $x^+ = \max(0, x)$.

\begin{lemma}
  \label{upper-bound:lemma:vector-payoffs}
  При применении стратегии $\tauv^k$ в игре $\G(\p)$ игрок 2 в состоянии $s$
  гарантирует себе проигрыш не более
  \begin{gather*}
    h^s_n(\tauv^k) = \sum_{t=0}^{n-1} (k-s-t-1+\beta)^+, \; s \leqslant k, \quad
    h^s_n(\tauv^k) = \sum_{t=0}^{n-1} (s-k-t-\beta)^+, \; s \geqslant k.
  \end{gather*}
  Последовательность $\left\{ h^s_n(\tauv^k), \; n = \overline{1, \infty}
  \right\}$ монотонна, ограничена сверху и сходится к %
  $h^s_\infty(\tauv^k) = (s - k + 1 - 2\beta)(s-k)/2$.
\end{lemma}

Введем следующие обозначения для множества распределений на $S$ с заданным
математическим ожиданием
\begin{equation*}
  \Theta(x) = \left\{ \p' \in \Delta(S): \E \p' = x \right\}, \quad
  \Lambda(x, y) = \left\{ \p' \in \Delta(S): x < \E \p' \leqslant y \right\}.
\end{equation*}

\begin{theorem}
  \label{upper-bound:theorem}
  Выигрыш в игре $\G[\infty](\p)$ ограничен сверху функцией
  \begin{equation*}
    \High(\p) = \min_{k \in J} \sum_{s \in S} p_s  h^s_\infty(\tauv^k).
  \end{equation*}
  Функция $\High(\p)$ является кусочно-линейной с областями недифференцируемости
  $\Theta(k+\beta)$ и областями линейности $\Lambda(k - 1 + \beta, s + \beta)$
  при $k \in S$. Для распределений $\p$ таких, что $\E \p = k - 1 + \beta + \xi,
  \; \xi \in [0, 1)$ ее значение равно
  \begin{equation*}
    \High(\p) = \D \p + \beta(1-\beta) - \xi(1-\xi).
  \end{equation*}
  Стратегия $\tauv^*$, которая позволяет игроку 2 обеспечить данную оценку,
  состоит в применении $\tauv^k$ при $\p \in \Lambda(k - 1 + \beta, k + \beta)$.
\end{theorem}
Заметим, что как и в \cite{bib:pyanykh16}, в данном случае наблюдается сдвиг
областей линейности на $\beta$ относительно $\E \p$ в сравнении с результатами в
\cite{bib:domansky11}.

% Перейдем к описании стратегии игрока 1, которая гарантирует ему выигрыш не менее
% $\High(\p)$.

\appendix
\setcounter{secnumdepth}{0}
\section{Приложение}

\begin{proof}[Доказательство леммы \ref{upper-bound:lemma:vector-payoffs}]
  TODO
\end{proof}

\begin{proof}[Доказательство теоремы \ref{upper-bound:theorem}]
  TODO
\end{proof}

\begin{thebibliography}{99}
% \bibitem{bib:demeyer02}%
%   De Meyer B., Saley H. \emph{On the strategic origin of Brownian motion in
%     finance} // International Journal of Game Theory. 2002. V. 31. P. 285--319.

\bibitem{bib:domansky07}%
  Domansky V. \emph{Repeated games with asymmetric information and random price
    fluctuations at finance markets} // International Journal of Game Theory.
  2007. V. 36(2). P. 241--257.

\bibitem{bib:aumann}%
  Aumann R.J., Maschler M.B. \emph{Repeated Games with Incomplete Information}.
  The MIT Press, Cambridge, London.

\bibitem{bib:sandomirskaya12}%
  Сандомирская М.С., Доманский В.К. \emph{Решение одношаговой игры биржевых
    торгов с неполной информацией} // Математическая теория игр и ее приложения.
  2012. 4. №1. С. 32-54.

\bibitem{bib:kreps09}%
  Крепс В.Л. \emph{Повторяющиеся игры, моделирующие биржевые торги, и возвратные
    последовательности} // Известия РАН. Теория и системы управления. 2009. № 4.
  С. 109--120.

\bibitem{bib:domansky11}%
  Доманский В.К., Крепс В.Л. \emph{Теоретико-игровая модель биржевых торгов:
    стратегические аспекты формирования цен на фондовых рынках} // Журнал Новой
  экономической ассоциации. 2011. Вып. 11. C. 39–-62.

\bibitem{bib:chatterjee83}%
  Chatterjee K., Samuelson W. \emph{Bargaining under Incomplete Information} //
  Operations Research. 1983. V. 31. N. 5. P. 835--851.

% \bibitem{bib:pyanykh14}%
%   Пьяных А.И. \emph{Об одной модификации модели биржевых торгов с инсайдером} //
%   Математическая теория игр и её приложения. 2014. 6. № 4. C. 68--84.

\bibitem{bib:pyanykh16}%
  Пьяных А.И. \textit{Многошаговая модель биржевых торгов с асимметричной
    информацией и элементами переговоров} // Вестн. Моск. ун-та. Сер.15. Вычисл.
  матем. и киберн. 2016. №1. С. 34—40.
\end{thebibliography}

\end{document}
